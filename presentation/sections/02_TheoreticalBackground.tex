\begin{frame}
	\sectionpage
\end{frame}

% \begin{frame}{Differential Inclusions (DI's)}
% 	Uncertain or discontinuous systems are more appropriately described by \textbf{Differential Inclusions} (DI) $\dot{x} \in F(t,x)$ than by Differential Equations (DE).
% 	% \vspace{0.5cm}
% 	% A \textbf{solution} of these DI's is any function $x(t)$ such that, defined in some interval $I\subseteq [0,\infty)$, which is absolutely continuous on each compact subinterval of $I$ and such that $\dot{x} \in F(t,x(t))$ almost everywhere in $I$.
% 	\begin{block}{Filippov Construction}
% 		For a discontinuous DE $\dot{x} = f(t,x)$, the Filippov DI is constructed as:
% 		$$F(t,x) = \bigcap_{\delta > 0} \bigcap_{\mu(S) = 0} \text{co}\{f(B(x,\delta) \setminus S)\}$$
% 	\end{block}
% 	where $B(x,\delta)$ is the ball of radius $\delta$ centered at $x$, and $\mu(S) = 0$ means that the set $S$ has Lebesgue measure zero, and $\text{co}$ denotes the convex hull. \cite{Filippov1988}
% \end{frame}

% \begin{frame}{Differential Inclusions (DI's)}
% 	\begin{block}{Standard Assumptions}
% 		$F(t,x)$ satisfies the standard assumptions if:
% 		\begin{itemize}
% 			\item[\textbf{(A1)}] $F(t,x)$ is nonempty, compact, convex
% 			\item[\textbf{(A2)}] Upper semicontinuous in $x$
% 			\item[\textbf{(A3)}] Measurable in $t$
% 			\item[\textbf{(A4)}] Locally bounded
% 		\end{itemize}
% 	\end{block}

% 	\begin{alertblock}{Existence}
% 		If (A1)-(A4) hold, then there exists at least one solution for each initial condition $(t_0, x_0)$.
% 	\end{alertblock}

% \end{frame}


% \begin{frame}{Homogeneity}
%     For a given vector $x=[x_1, x_2, \ldots, x_n]^T \in \mathbb{R}^n$ and for every $\epsilon > 0$, the \textbf{dilation operator} is defined as:
%     $$
%         \delta_{\epsilon}^r x \equiv [\epsilon^{r_1} x_1, \epsilon^{r_2} x_2, \ldots, \epsilon^{r_n} x_n]^T,
%     $$
%     where $r_i > 0$ are the \textbf{weights} of the coordinates, and let 
% \end{frame}

\begin{frame}{Differential Inclusions}

	\textbf{Why Differential Inclusions?}

	Uncertain or discontinuous systems are more appropriately described by Differential Inclusions (DI)
	$$\dot{x} \in F(t,x)$$
	than by Differential Equations (DE).

	\vspace{0.5cm}
	\textbf{Solution of a DI:}

	A solution of the DI $\dot{x} \in F(t,x)$ is any function $x(t)$, defined in some interval $I \subseteq [0,\infty)$, which is:
	\begin{itemize}
		\item Absolutely continuous on each compact subinterval of $I$
		\item Satisfies $\dot{x}(t) \in F(t,x(t))$ almost everywhere on $I$
	\end{itemize}

	For a discontinuous DE $\dot{x} = f(t,x)$, the function $x(t)$ is a \alert{generalized solution} if and only if it is a solution of the associated DI $\dot{x} \in F(t,x)$.

\end{frame}

\begin{frame}{Filippov Differential Inclusions}
  \small
	We consider the DI $\dot{x} \in F(t,x)$ associated to $\dot{x} = f(t,x)$ as given by \textbf{A.F. Filippov's approach} - the \alert{Filippov DI} with \alert{Filippov solutions} \cite{Filippov1988}.

	\vspace{0.5cm}
	\textbf{Standard Assumptions:}

	The multivalued map $F(t,x)$ satisfies the standard assumptions if:

	\begin{itemize}
		\item[\textbf{(H1)}] $F(t,x)$ is nonempty, compact, convex subset of $\mathbb{R}^n$
		\item[\textbf{(H2)}] $F(t,x)$ is upper semi-continuous as a set valued map of $x$
		\item[\textbf{(H3)}] $F(t,x)$ is Lebesgue measurable as a set valued map of $t$
		\item[\textbf{(H4)}] $F(t,x)$ is locally bounded
	\end{itemize}

	\vspace{0.3cm}
	\alert{Existence Theorem:} If $F(t,x)$ satisfies (H1)-(H4), then for each pair $(t_0,x_0) \in [0,\infty) \times \mathbb{R}^n$ there exists at least one solution $x(t)$ with $x(t_0) = x_0$.
\end{frame}

\begin{frame}{Homogeneity - Dilation and Functions}

	\textbf{Dilation Operator:}

	For $x = [x_1, \ldots, x_n]^\top$ and $\lambda > 0$:
	$$\Lambda_r^\lambda x := [\lambda^{r_1} x_1, \ldots, \lambda^{r_n} x_n]^\top$$
	where $r = [r_1, \ldots, r_n]^\top$ with $r_i > 0$ are the \textbf{weights} of the coordinates.

	\vspace{0.5cm}
	\textbf{$r$-Homogeneous Function:}

	A function $V: \mathbb{R}^n \to \mathbb{R}$ is called \alert{$r$-homogeneous of degree $l \in \mathbb{R}$} if:
	$$V(\Lambda_r^\lambda x) = \lambda^l V(x) \quad \text{for every } \lambda > 0$$

	\textbf{$r$-Homogeneous Vector Field:}

	A vector field $f: \mathbb{R}^n \to \mathbb{R}^n$ is called \alert{$r$-homogeneous of degree $l$} if:
	$$f(\Lambda_r^\lambda x) = \lambda^l \Lambda_r^\lambda f(x)$$

\end{frame}

\begin{frame}{Homogeneity - Norm and Systems}

	\textbf{Homogeneous Norm:}

	Given a vector $r$ and dilation $\Lambda_r^\lambda x$, the homogeneous norm is defined by:
	$$\|x\|_{r,p} := \left(\sum_{i=1}^n |x_i|^{p/r_i}\right)^{1/p}, \quad \forall x \in \mathbb{R}^n$$
	for any $p \geq 1$.

	\vspace{0.5cm}
	\textbf{Homogeneous System:}

	A system is called \alert{homogeneous} if its vector field (or vector-set field) is $r$-homogeneous of some degree.

	\vspace{0.3cm}
	A vector-set field $F(x) \subset \mathbb{R}^n$ is \alert{$r$-homogeneous of degree $l$} if:
	$$F(\Lambda_r^\lambda x) = \lambda^l \Lambda_r^\lambda F(x)$$

\end{frame}

\begin{frame}{Homogeneous Differential Inclusions}

	\textbf{Homogeneous DI:}

	A DI $\dot{x} \in F(x)$ is $r$-homogeneous of degree $l$ if the vector-set field $F(x)$ satisfies:
	$$F(\Lambda_r^\lambda x) = \lambda^l \Lambda_r^\lambda F(x), \quad \forall \lambda > 0$$

	\vspace{0.5cm}
	\textbf{Key Property - Local to Global:}

	For homogeneous systems of degree $l < 0$, \alert{local stability implies global stability}. This remarkable property allows homogeneity to "extend" local properties to global ones.

	\vspace{0.5cm}
	\textbf{Lyapunov Analysis:}

	For homogeneous DI, if there exists a homogeneous Lyapunov function $V(x)$ such that:
	$$\frac{\partial V}{\partial x} \nu \leq -c(\|x\|) \quad \text{for all } \nu \in F(x)$$
	then the origin is Uniformly Globally Asymptotically Stable (UGAS).

\end{frame}

\begin{frame}{Finite-Time Stability}

	\textbf{Global Uniform Finite-Time Stability (GUFTS):}

	A DI $\dot{x} \in F(x)$ is GUFTS at 0 if:
	\begin{itemize}
		\item $x(t) = 0$ is a Lyapunov-stable solution
		\item For any $R > 0$ there exists $T > 0$ such that any trajectory starting within $\|x\| < R$ reaches zero in time $T$
	\end{itemize}

	\vspace{0.5cm}
	\alert{Fundamental Result:} For $r$-homogeneous DI's of degree $l < 0$:
	$$\text{Local asymptotic stability} \Rightarrow \text{Global finite-time stability}$$

	\vspace{0.5cm}

	\textbf{Application:} Homogeneous discontinuous controllers achieve:
	\begin{itemize}
		\item Finite-time convergence
		\item Robustness against perturbations
		\item Global stability from local analysis
	\end{itemize}
\end{frame}
